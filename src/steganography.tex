%! Author = kucera-lukas
%! Date = 4/13/22

\section{Steganografie}\label{sec:steganografie}

\subsection{Definice}\label{subsec:definice}
Steganografie je vědní disciplína (podobor kryptografie) zabývající se utajením
komunikace prostřednictvím ukrytí zprávy.
Zpráva je ukryta tak, aby si pozorovatel neuvědomil,
že komunikace vůbec probíhá.\cite{wiki:steganografie}

\subsection{Implementace}\label{subsec:implementace}
Uzivateli

\subsection{Zakodovani}\label{subsec:zakodovani-dat}
Na adrese \url{https://stegoer.netlify.app/encode} může uživatel nahrát svůj
obrázek a zakódovat do něj informaci.

\begin{subsubsection}{Textovy editor}\label{subsubsec:textovy-editor}
V horní části stránky se nachází textový editor.
Do něj uživatel zadá data, která mají byt skryta v obrázku.
Editor umožnuje mnoho možností pro formátování textu.

\begin{figure}
    \centering
    \includegraphics[scale=0.5]{assets/images/encode-editor}
    \caption{Editor textu pro data k zakódováni}\label{fig:editor-textu}
\end{figure}

\end{subsubsection}

\begin{subsubsection}{Konfigurace}\label{subsubsec:enc-konfigurace}

Konfigurace je dostupná pouze pro přihlášené uživatele.

V sekci \texttt{Advanced Configuration} je dostupné nastaveni parametrů
pro zakódování.

\begin{enumerate}
    \item Vlastní klíč pro zašifrování dat.
    \item Počet nejméně významných bitů, které mají být použity.
    \item Určení jaké barevné složky mají být změněny.
    \item Zda-li mají být data rozprostřena po obrázku rovnoměrně.
\end{enumerate}

\end{subsubsection}

\begin{figure}
    \centering
    \includegraphics[scale=0.5]{assets/images/encode-configuration}
    \caption{Konfigurace zakódování}\label{fig:konfigurace-zakodovani}
\end{figure}

\begin{subsubsection}{Nahrání obrázku}\label{subsubsec:enc-nahrani-obrazku}

Na stránce nalezneme pole do kterého lze nahrát obrázek.
Do něj budou informace zakódovány.

\end{subsubsection}

\begin{subsubsection}{Potvrzení}\label{subsubsec:enc-potvrzeni}

Po stisknutí tlačítka \texttt{Encode} proběhne kontrola vložených hodnot.
Pokud je nějaký problém nalezen aplikace uživatele upozorní.

Dále aplikace pošle požadavek s informací, konfigurací a
obrázkem na server.

Následně server zprávu zašifruje a rozdělí na jednotlivé bity a zapíše je do
obrázku.

Pokud vše proběhne v pořádku klient dostane zpět odpověd s tímto upraveným
obrázkem.

Dále se uživateli nabídne možnost si obrázek stáhnout.

\end{subsubsection}

\subsection{Dekódování}\label{subsec:dekodovani-dat}
Na adrese \url{https://stegoer.netlify.app/decode} může uživatel nahrát svůj
obrázek a dekódovat z něj data, která tam v minulosti zakódoval.

\begin{subsubsection}{Konfigurace}\label{subsubsec:dec-konfigurace}

V sekci \texttt{Advanced Configuration}, která je opět dostupná pouze
přihlášeným uživatelům, je možné specifikovat vlastní klíč, který má být použit
k odšifrování zprávy.

\end{subsubsection}

\begin{subsubsection}{Nahráni obrázku}\label{subsubsec:dec-nahrani-obrazku}

Na stránce opět nalezneme pole do kterého lze nahrát obrázek.
Z tohotu obrázku se aplikace pokusí zakódovanou informaci dekódovat.

\end{subsubsection}

\begin{subsubsection}{Potvrzení}\label{subsubsec:dec-potvrzeni}

Po stisknutí tlačítka \texttt{Decode} proběhne kontrola vložených hodnot.
Pokud je vše v pořádku aplikace pošle požadavek s obrázkem a konfiguraci na
server.

Následně server obrázek zpracuje a pokusí se zakódovanou informaci odšifrovat.
Pokud vše proběhne v pořádku klient dostane zpět odpověd s odšifrovanou
informací.

Na konci stránky se pak tato informace zobrazí.
Aplikace také informaci uživateli zkopíruje do schránky, aby s ní mohl dále
pracovat.

\end{subsubsection}
